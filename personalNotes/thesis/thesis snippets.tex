\documentclass[12pt,a4paper]{article}
\usepackage[ngerman, english]{babel}
\usepackage[utf8]{inputenc}
\usepackage[unicode=true,bookmarks=false,bookmarksopen=true]{hyperref}

\usepackage{xcolor}
\usepackage{graphicx}
\usepackage{tikz}

\usepackage{listings}

\def\checkmark{\tikz\fill[scale=0.4](0,.35) -- (.25,0) -- (1,.7) -- (.25,.15) -- cycle;}

\definecolor{pGreen}{rgb}{0.44, 0.71, 0}
\definecolor{nRed}{rgb}{0.74, 0, 0}

\title{ORES Custom Documentation - Thesis Snippets}
\author{Tom Gülenman}
\date{}
\begin{document}
\tableofcontents
\section{Introduction}
ORES\footnote{\url{https://ores.wikimedia.org/}} stands for Objective Revision Evaluation Service and is a machine learning-powered webservice developed by the Wikimedia Scoring Platform team\footnote{\url{https://www.mediawiki.org/wiki/Wikimedia_Scoring_Platform_team}}. As of now, it offers a restful API\footnote{https://ores.wikimedia.org/v3/} and a very basic user interface\footnote{https://ores.wmflabs.org/ui/} allowing users to retrieve scoring information about edits across a multitude of wikis.
%TODO edits = broad term = edits, new article creations etc.
%TODO which wikis?
%TODO scores have been determined by ORES' machine learning algorithms 
\end{document}